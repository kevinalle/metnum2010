\documentclass[12pt,titlepage]{article}
\usepackage[spanish]{babel}
\usepackage[utf8]{inputenc}
\usepackage{amsmath}
\usepackage{amssymb}
\usepackage{graphicx}
\usepackage{caratulaMetNum}

\usepackage{color}
\usepackage{url}
\definecolor{lnk}{rgb}{0,0,0.4}
\usepackage[colorlinks=true,linkcolor=lnk,citecolor=blue,urlcolor=blue]{hyperref}

\newcommand{\func}[2]{\texttt{#1}(#2) :}
\newcommand{\tab}{\hspace*{2em}}
\newcommand{\FOR}{\textbf{for }}
\newcommand{\TO}{\textbf{ to }}
\newcommand{\IF}{\textbf{if }}
\newcommand{\WHILE}{\textbf{while }}
\newcommand{\THEN}{\textbf{then }}
\newcommand{\ELSE}{\textbf{else }}
\newcommand{\RET}{\textbf{return }}
\newcommand{\MOD}{\textbf{ \% }}
\newcommand{\OR}{\textbf{ or }}
\newcommand{\NOT}{\textbf{ not }}
\newcommand{\tOde}[1]{\tab \small{\mathcal{O}($#1$)}}
\newcommand{\Ode}[1]{\ensuremath{\small{\mathcal{O}\left(#1\right)}}}
\newcommand{\VSP}{\vspace*{3em}}
\newcommand{\Pa}{\vspace{5mm}}
\newenvironment{pseudo}{\noindent\begin{tabular}{ll}}{\end{tabular}\VSP}

\newenvironment{while}{\WHILE \\ \setlength{\leftmargin}{0em} }{}

\newcommand{\iif}{\Leftrightarrow}
\newcommand{\gra}[1]{\noindent\includegraphics[scale=.70]{#1}\\}
\newcommand{\gram}[1]{\noindent\includegraphics[scale=.50]{#1}}
\newcommand{\dirmail}[1]{\normalsize{\texttt{#1}}}
\newenvironment{usection}[1]{\newpage\begin{section}*{#1}	\addcontentsline{toc}{section}{#1}}{\end{section}}
\newenvironment{usubsection}[1]{\begin{subsection}*{#1}	\addcontentsline{toc}{subsection}{#1}}{\end{subsection}}

%\title{{\sc\normalsize Métodos Numéricos}\\{\bf Trabajo Práctico Nº1}}
%\author{\begin{tabular}{lcr}Pablo Herrero & LU & \dirmail{pablodherrero@gmail.com}\\Thomas Fischer & 489/08 & \dirmail{tfischer@dc.uba.ar}\\Kevin Allekotte & 490/08 & \dirmail{kevinalle@gmail.com} \end{tabular}}
%\date{\VSP \normalsize{Abril 2010}}
\begin{document}

\materia{Métodos Numéricos}
\titulo{Trabajo Práctico Nº1}
\subtitulo{``Perdidos'' en el Pacífico}
%\grupo{Grupo x}
\integrante{Pablo Herrero}{332/07}{pablodherrero@gmail.com}
\integrante{Thomas Fischer}{489/08}{tfischer@dc.uba.ar}
\integrante{Kevin Allekotte}{490/08}{kevinalle@gmail.com}

\abstracto{
	Este trabajo pretende hacer una comparación sobre el desempeño y la precisión de la aritmética flotante de diversos entornos y
	mostrar en los resultados  lo que pueden producir pequeños errores de cálculo.
	Para ello trabajamos sobre la \textit{Recurrencia de Müller} detallada en el Apéndice A,
	sobre la cuál sabemos que converge y conocemos una ecuación cerrada para sus términos,
	lo que nos permite calcular el límite exacto analíticamente.
	Las pruebas consisten en comparar el límite exacto de la sucesión con el límite obtenido por la sucesión de recurrencia
	programada en distintos entornos y lenguajes.
}

\palabraClave{perdidos}
\palabraClave{pacifico}

\begin{titlepage}
\maketitle
\end{titlepage}
\tableofcontents
\newpage

	\begin{usection}{Introducción teorica}

		Cuando trabajamos con números reales en una computadora,
		nos encontramos con algunos factores limitantes para nuestros cálculos
		como lo son la aritmética finita y
		también el tipo de representación que se nos provee para trabajar. 	\\
		Esto surge como una consecuencia de la necesidad
		de una memoria física finita de la computadora
		en contraste con la precisión infinita
		que requieren la mayoría de los números reales.
\\
		Luego, al intentar representar números en una computadora,
		cuya precisión va mas allá de la otorgada por la misma,
		surgen en la representación del número pequeños errores,
		los cuáles, como muestra este trabajo, no son despreciables
		al realizarse ciertos tipos de cálculos.
\\
		Cada uno de estos sistemas puede tener una precisión muy distinta,
		lo que produce una diferencia en los errores cometidos al representar
		números en cada uno.
\\
		Además existen distintos tipos de precisión a analizar, 
		como por ejemplo la resolución o granularidad,
		o los limites superiores e inferiores,
		tanto de la parte entera como de la parte decimal.
\\
		La representación que elijamos depende totalmente
		de los tipos de cálculo que vallamos a realizar
		y el tipo de precisión que necesitemos
		para obtener un resultado razonable.
\\
		Existen múltiples estándares de representación para los números reales,
		de los cuáles tal vez el más difundido para números reales
		sea la norma \texttt{IEEE - 754}.

	\end{usection}
	
	\begin{usection}{Desarrollo}

		\begin{usubsection}{Análisis Teórico de la sucesión}

			En primer lugar, es necesario un análisis
			de la fórmula cerrada de la \textit{Recurrencia de Müller},
			para encontrar el límite al cuál converge.

			La formula cerrada para los términos de la \textit{Recurrencia de Müller} esta dada por

			\begin{displaymath}
				x_n \ =\ \frac{\alpha 3^{n+1} + \beta 5^{n+1} + \gamma 100^{n+1}}{\alpha 3^n + \beta 5^n + \gamma 100^n}
			\end{displaymath}
\\
			por lo cuál el límite esta dado por $\lim_{n\to\infty} x_n$. Separando en caso, calculamos los posibles límites de la siguiente manera.

			\begin{itemize}

				\item para $\gamma \neq 0$

					\begin{eqnarray*}
						\lim_{n\to\infty} \frac{\gamma 100^{n+1}}{\alpha 3^n + \beta 5^n + \gamma 100^n} &
						\leq \lim_{n\to\infty} x_n &
						\leq \lim_{n\to\infty} \frac{\alpha 3^{n+1} + \beta 5^{n+1} + \gamma 100^{n+1}}{\gamma 100^n}
					\\
						\lim_{n\to\infty} \frac{\gamma 100}{\alpha{(\frac{3}{100})}^n + \beta{(\frac{5}{100})}^n + \gamma} &
						\leq \lim_{n\to\infty} x_n &
						\leq \lim_{n\to\infty} \frac{\alpha 3^{n+1}}{\gamma 100^n} + \frac{\beta 5^{n+1}}{\gamma 100^n} + \frac{\gamma 100^{n+1}}{\gamma 100^n}
					\\
						\lim_{n\to\infty} 100 &
						\leq \lim_{n\to\infty} x_n &
						\leq \lim_{n\to\infty} \frac{3\alpha}{\gamma} {(\frac{3}{100})}^n + \lim_{n\to\infty} \frac{5\beta}{\gamma} {(\frac{5}{100})}^n + \lim_{n\to\infty} 100
					\\						
						100 &\leq \lim_{n\to\infty} x_n &\leq 100
					\\
						\Rightarrow &\lim_{n\to\infty} x_n &= 100
					\end{eqnarray*}

				\item para $\gamma = 0$ y $\beta \neq 0$

					\begin{eqnarray*}
						\lim_{n\to\infty} \frac{\beta 5^{n+1}}{\alpha 3^n + \beta 5^n} &
						\leq \lim_{n\to\infty} \frac{\alpha 3^{n+1} + \beta 5^{n+1}}{\alpha 3^n + \beta 5^n} &
						\leq \lim_{n\to\infty} \frac{\alpha 3^{n+1} + \beta 5^{n+1}}{\beta 5^n}
					\\
						\lim_{n\to\infty} \frac{\beta 5}{\alpha{(\frac{3}{5})}^n + \beta} &
						\leq \lim_{n\to\infty} x_n &
						\leq \lim_{n\to\infty} \frac{\alpha 3^{n+1}}{\beta 5^n} + \frac{\beta 5^{n+1}}{\beta 5^n}
					\\
						\lim_{n\to\infty} 5 &
						\leq \lim_{n\to\infty} x_n &
						\leq \lim_{n\to\infty} \frac{3\alpha}{\beta} {(\frac{3}{5})}^n + \lim_{n\to\infty} 5
					\\
						5 &\leq \lim_{n\to\infty} x_n &\leq 5
					\\
						& \Rightarrow \lim_{n\to\infty} x_n &= 5
					\end{eqnarray*}

				\item para $\gamma = 0$, $\beta = 0$ y $\alpha \neq 0$

					$$
						\lim_{n\to\infty} \frac{\alpha 3^{n+1}}{\alpha 3^n}
						= \lim_{n\to\infty} 3
						= 3
					$$
\\
				\item para $\gamma = \beta = \alpha = 0$ la función no esta definida,
				y por lo tanto no nos interesa, porque suponemos que para todo $x0$, $x1$ existen
				$\alpha$, $\beta$ y $\gamma$ que cumplen con la fórmula cerrada
				de tal manera que alguno de ellos es distinto de $0$.

			\end{itemize}
 
			Luego queremos ver para que casos de entrada, o valores de $x0$ y $x1$,
			se dan las distintas opciones para $\alpha$, $\beta$ y $\gamma$.

			Consideramos que si $\gamma$ puede ser $0$ entonces lo es, y solamente en caso contrario $\gamma!=0$.

			\begin{itemize}

				\item para $\gamma = 0$

					Despejamos la fórmula cerrada para $x0$ y $x1$ y restando las ecuaciones obtenemos la igualdad

					\begin{displaymath}
						b( \frac{5-x_0}{x_0-3} - \frac{5}{3} \frac{5-x_1}{x_1-3} ) = 0
					\end{displaymath}

					de la cuál podemos deducir que $x0(x1-8) = -15$.

					Dentro de este caso existe otro caso especial en el cuál
					$\beta = 0$ y $\alpha \neq 0$. Podemos ver que este caso solo aparece
					cuando $x0 = x1 = 3$.				

				\item El resto de las entradas cae en el caso $\gamma \neq 0$

%					Para este caso, divido tanto dividendo como divisor
%					de la fórmula cerrada por $\gamma$,
%					generando así una nueva expresión de la fórmula,
%					esta vez en funcion de 2 variables
%					$\alpha' = \frac{\alpha}{\gamma}$ y $\beta'= \frac{\beta}{\gamma}$

%					\begin{displaymath}
%						x_n \ =\ \frac{\alpha' 3^{n+1} + \beta' 5^{n+1} + 100^{n+1}}{\alpha' 3^n + \beta' 5^n + 100^n}
%					\end{displaymath}

			\end{itemize}
 
			Luego quedan resueltos los límitesde la \textit{Recurrencia de Müller}
			en función de las distintas entradas de la siguiente manera:
 
 			\begin{itemize}
				\item Si $x0 = x1 = 3$ entonces $\lim_{n\to\infty} x_n = 3$
				\item Si $x0(x1-8) = -15$ entonces $\lim_{n\to\infty} x_n = 5$
				\item Sino $\lim_{n\to\infty} x_n = 100$
 			\end{itemize} 			
 
		\end{usubsection}

		\begin{usubsection}{Implementación}

			Luego el trabajo consistió en implementar
			la \textit{Recurrencia de Müller} en distintos lenguajes y plataformas,
			y calcular hasta un n-ésimo término la sucesión
			de tal manera que $|x_n-x_{n-1}|<\epsilon$
			donde $\epsilon$ es una diferencia entre términos sucesivos
			a partir de la cuál consideramos que la recurrencia
			ya ha alcanzado el límite. \\
\\
			Los resultados de ejecutar estos programas son comparados
			con los resultados teóricos de límite obtenidos,
			explicados en la seccion precedente.
			Así podemos obtener un valor del error relativo
			cometido por el algoritmo por consecuencia de las limitaciones
			de representación aritmética de las computadoras. \\
\\
			Los resultados obtenidos son analizados en la sección siguiente.

		\end{usubsection}		

	\end{usection}
	
	\begin{usection}{Resultados}
		\gra{lim5.png}
		
		\gra{fig_2_8_20.png}
		
		\gra{fig_4_4-25_20.png}
		
		\gra{fig_4_4-25_10.png}
		
		\gra{fig_5_5_20.png}

	\end{usection}
	
	\begin{usection}{Discusión}
	\end{usection}
	
	\begin{usection}{Conclusiones}
	\end{usection}
	
	\begin{usection}{Apéndices}
		\begin{usubsection}{Apéndice A: Enunciado}
			\begin{centering}
\bf Laboratorio de M\'etodos Num\'ericos - Primer cuatrimestre 2010 \\
\bf Trabajo Pr\'actico N\'umero 3: <El TP del Mundial! \\
\end{centering}

\vskip 25pt
\hrule
\vskip 11pt

\textbf{Introducci\'on}

Nos encontramos en la m\'axima cita del f\'utbol mundial...~de robots.
Uno de los problemas m\'as importantes a resolver en este contexto es
predecir con la mayor anticipaci\'on posible la posici\'on futura de la 
pelota en funci\'on de su posici\'on en el pasado reciente. Sobre la base
de estas predicciones se coordinan los movimientos de los jugadores de
campo y, en el caso que nos ocupa ahora, la posici\'on del arquero cuando
existe peligro de gol.

Cuando la pelota se dirige hacia nuestro arco, es muy importante que
ubiquemos el arquero en la posici\'on exacta en la que la pelota 
cruzar\'a la l\'\i nea de gol, de manera que pueda interceptarla y evitar
la ca\'\i da de nuestra valla. El sistema de control de cada equipo
suministra informaci\'on en pasos discretos. En cada paso nuestras
c\'amaras de video determinan la posici\'on de la pelota, y debemos
indicarle la acci\'on a seguir al arquero: quedarse quieto, moverse 
hacia la izquierda o moverse hacia la derecha.

Los postes del arco est\'an ubicados en las coordenadas (-1,0) y (1,0),
y la l\'\i nea de gol es el segmento entre estos dos puntos. Se marca
un gol cuando la pelota cruza este segmento. Vistos desde arriba, la pelota
es un c\'\i rculo de 0.10 de radio y el arquero 
se representa mediante un segmento paralelo a la l\'\i nea de gol de 0.10 
de longitud y ubicado sobre la misma.
% es un jugador cuadrado de 0.10 de lado, 
% cuyo punto central est\'a ubicado sobre la l\'\i nea de gol. 
Inicialmente el punto central del arquero se encuentra en la
posici\'on (0,0), y en cada paso se le indica al arquero qu\'e acci\'on
debe tomar. Si se le indica un movimiento hacia alguno de los lados
(izquierda o derecha) y en el paso anterior estaba quieto o se estaba
moviendo hacia ese mismo lado, entonces el arquero se mueve 0.05 en
la direcci\'on indicada. Por el contrario, si en el paso anterior se
hab\'\i a indicado un movimiento en la direcci\'on opuesta, entonces
el arquero se queda quieto durante el paso actual.

Un problema fundamental que debe enfrentarse es la presencia de ruido
en las mediciones de la posici\'on de la pelota. El sistema de visi\'on
est\'a sujeto a vibraciones, golpes y errores de captura de datos,
que hacen que las mediciones de la pelota sufran errores, e incluso 
registren posiciones irreales (es un efecto muy com\'un que la pelota
``desaparezca'' en un cuadro y vuelva a aparecer en el cuadro siguiente).
Por otra parte, la pelota no siempre viaja hacia el arco en l\'\i nea
recta sino que puede describir curvas m\'as o menos complicadas,
dependiendo del ``efecto'' dado por el jugador al momento de impactar
la pelota y de posibles curvaturas en la superficie del campo de juego.

\textbf{Enunciado}

El objetivo del trabajo pr\'actico es implementar un programa que
tome como datos las posiciones sucesivas de la pelota y que determine
en cada paso qu\'e debe hacer el arquero para evitar el gol. Se deben
tomar las mediciones con la posici\'on de la pelota de un archivo
de entrada, que tiene en cada l\'inea el n\'umero de medici\'on,
la posici\'on $x$ y la posici\'on $y$ de la pelota, separados
por espacios. La \'ultima l\'\i nea del archivo tiene -1 como
primer dato, indicando el fin de las mediciones.

En cada paso, el programa debe determinar qu\'e acci\'on debe tomar
el arquero, escribiendo la decisi\'on correspondiente en un archivo
de salida. Cada l\'\i nea de este archivo debe tener el n\'umero
de medici\'on y la acci\'on del arquero (0: quedarse quieto, 1: izquierda,
2: derecha), separados por espacio. El programa debe tomar por
l\'\i nea de comandos el nombre del archivo de entrada y el nombre
del archivo de salida. Dado que estamos simulando
la decisi\'on en tiempo real, para generar la acci\'on correspondiente
a una medici\'on el programa solamente puede usar la informaci\'on
de esa medici\'on y las anteriores (es decir, no se puede consultar
lo que suceder\'a en el futuro para tomar las decisiones).

Las instrucciones al arquero deben estar basadas en un mecanismo de
predicci\'on de la posici\'on futura de la pelota. Esta predicci\'on
se debe realizar sobre la base de alg\'un m\'etodo num\'erico visto
en la materia, o alguna variaci\'on de los temas vistos en clase.
Sugerimos consultar con los docentes del laboratorio para validar
los enfoques que propongan implementar.

Se adjunta a este enunciado un programa simulador de los tiros al
arco, junto con algunos archivos de prueba para testear el formato
de los archivos de entrada y salida. Todos los programas participar\'an
de un campeonato mundial de arqueros, y el grupo cuyo arquero logre
atajar la mayor cantidad de tiros al arco se har\'a acreedor a la
copa ``Laboratorio de M\'etodos Num\'ericos'' al mejor guardavallas del
mundial.

\textbf{Preguntas adicionales}

\begin{enumerate}
\item >En qu\'e minuto del partido Argentina-Uni\'on Sovi\'etica del mundial
Italia '90 se lesion\'o Nery Pumpido, arquero de la selecci\'on
argentina?

\item >C\'omo se llamaba el \'arbitro del partido Argentina-Francia en
el mundial Argentina '78?

\item >Cu\'antos jugadores fueron expulsados en el mundial Italia '90
por derribar a Claudio Paul Caniggia?

\item >Cu\'antos pases hizo la selecci\'on argentina antes del gol de Maradona
ante Grecia en el mundial EEUU 94?

\item >Cu\'antos minutos jug\'o Ricardo Bochini en la semifinal del
mundial M\'exico '86?

\item >C\'omo se llamaba el arquero suplente de la selecci\'on argentina
en la final del mundial Uruguay '30?
\end{enumerate}

\vskip 15pt

\hrule

\vskip 11pt

Fecha de entrega: Viernes 25 de Junio

		\end{usubsection}
		
		\begin{usubsection}{Apéndice B: Codigos Fuente}
			
		\end{usubsection}
	\end{usection}
	
	\begin{usection}{Referencias}
	\end{usection}

\end{document}

