\documentclass[12pt,titlepage]{article}
\usepackage[spanish]{babel}
\usepackage[utf8]{inputenc}
\usepackage{amsmath}
\usepackage{amssymb}
\usepackage{graphicx}
\usepackage{caratulaMetNum}
\usepackage{float}

\usepackage{color}
\usepackage{url}
\definecolor{lnk}{rgb}{0,0,0.4}
\usepackage[colorlinks=true,linkcolor=lnk,citecolor=blue,urlcolor=blue]{hyperref}

\newcommand{\func}[2]{\texttt{#1}(#2) :}
\newcommand{\tab}{\hspace*{2em}}
\newcommand{\FOR}{\textbf{for }}
\newcommand{\TO}{\textbf{ to }}
\newcommand{\IF}{\textbf{if }}
\newcommand{\WHILE}{\textbf{while }}
\newcommand{\THEN}{\textbf{then }}
\newcommand{\ELSE}{\textbf{else }}
\newcommand{\RET}{\textbf{return }}
\newcommand{\MOD}{\textbf{ \% }}
\newcommand{\OR}{\textbf{ or }}
\newcommand{\NOT}{\textbf{ not }}
\newcommand{\tOde}[1]{\tab \small{\mathcal{O}($#1$)}}
\newcommand{\Ode}[1]{\ensuremath{\small{\mathcal{O}\left(#1\right)}}}
\newcommand{\VSP}{\vspace*{3em}}
\newcommand{\Pa}{\vspace{5mm}}
\newenvironment{pseudo}{\noindent\begin{tabular}{ll}}{\end{tabular}\VSP}

\newenvironment{while}{\WHILE \\ \setlength{\leftmargin}{0em} }{}

\newcommand{\iif}{\Leftrightarrow}
\newcommand{\gra}[1]{\noindent\includegraphics[scale=.70]{#1}\\}
\newcommand{\gras}[2]{\noindent\includegraphics[scale=#2]{#1}\\}
\newcommand{\gram}[1]{\noindent\includegraphics[scale=.50]{#1}}
\newcommand{\dirmail}[1]{\normalsize{\texttt{#1}}}
\newenvironment{usection}[1]{\newpage\begin{section}*{#1}	\addcontentsline{toc}{section}{#1}}{\end{section}}
\newenvironment{usubsection}[1]{\begin{subsection}*{#1}	\addcontentsline{toc}{subsection}{#1}}{\end{subsection}}

\newcommand{\superref}[1]{\textsuperscript{\ref{#1}}}

%\title{{\sc\normalsize Métodos Numéricos}\\{\bf Trabajo Práctico Nº1}}
%\author{\begin{tabular}{lcr}Pablo Herrero & LU & \dirmail{pablodherrero@gmail.com}\\Thomas Fischer & 489/08 & \dirmail{tfischer@dc.uba.ar}\\Kevin Allekotte & 490/08 & \dirmail{kevinalle@gmail.com} \end{tabular}}
%\date{\VSP \normalsize{Abril 2010}}
\begin{document}

\materia{Métodos Numéricos}
\titulo{Trabajo Práctico Nº1}
\subtitulo{``Perdidos'' en el Pacífico}
%\grupo{Grupo x}
\integrante{Pablo Herrero}{332/07}{pablodherrero@gmail.com}
\integrante{Thomas Fischer}{489/08}{tfischer@dc.uba.ar}
\integrante{Kevin Allekotte}{490/08}{kevinalle@gmail.com}

\abstracto{
	Este trabajo pretende hacer una comparación sobre el desempeño y la precisión de la aritmética flotante de diversos entornos y
	mostrar en los resultados  lo que pueden producir pequeños errores de cálculo.
	Para ello trabajamos sobre la \textit{Recurrencia de Müller} detallada en el Apéndice A,
	sobre la cuál sabemos que converge y conocemos una ecuación cerrada para sus términos,
	lo que nos permite calcular el límite exacto analíticamente.
	Las pruebas consisten en comparar el límite exacto de la sucesión con el límite obtenido por la sucesión de recurrencia
	programada en distintos entornos y lenguajes.
}

\palabraClave{Aritmética Finita}
\palabraClave{Recurrencia de Müller}

\begin{titlepage}
\maketitle
\end{titlepage}
\tableofcontents
\newpage

	\begin{usection}{Introducción teorica}

		Cuando trabajamos con números reales en una computadora,
		nos encontramos con algunos factores limitantes para nuestros cálculos
		como lo son la aritmética finita y
		también el tipo de representación que se nos provee para trabajar. 	\\
		Esto surge como una consecuencia de la necesidad
		de una memoria física finita de la computadora
		en contraste con la precisión infinita
		que requieren la mayoría de los números reales.
\\
		Luego, al intentar representar números en una computadora,
		cuya precisión va mas allá de la otorgada por la misma,
		surgen en la representación del número pequeños errores,
		los cuáles, como muestra este trabajo, no son despreciables
		al realizarse ciertos tipos de cálculos.
\\
		Cada uno de estos sistemas puede tener una precisión muy distinta,
		lo que produce una diferencia en los errores cometidos al representar
		números en cada uno.
\\
		Además existen distintos tipos de precisión a analizar, 
		como por ejemplo la resolución o granularidad,
		o los limites superiores e inferiores,
		tanto de la parte entera como de la parte decimal.
\\
		La representación que elijamos depende totalmente
		de los tipos de cálculo que vallamos a realizar
		y el tipo de precisión que necesitemos
		para obtener un resultado razonable.
\\
		Existen múltiples estándares de representación para los números reales,
		de los cuáles tal vez el más difundido para números reales
		sea la norma \texttt{IEEE - 754}.

	\end{usection}
	
	\begin{usection}{Desarrollo}

		\begin{usubsection}{Análisis Teórico de la sucesión}

			En primer lugar, es necesario un análisis
			de la fórmula cerrada de la \textit{Recurrencia de Müller},
			para encontrar el límite al cual converge.

			La formula cerrada para los términos de la \textit{Recurrencia de Müller} esta dada por

			\begin{displaymath}
				x_n \ =\ \frac{\alpha 3^{n+1} + \beta 5^{n+1} + \gamma 100^{n+1}}{\alpha 3^n + \beta 5^n + \gamma 100^n}
			\end{displaymath}
\\
			por lo cuál el límite esta dado por $\lim_{n\to\infty} x_n$. Separando en casos, calculamos los posibles límites de la siguiente manera.

			\begin{itemize}

				\item para $\gamma \neq 0$

					\begin{eqnarray*}
						\lim_{n\to\infty} \frac{\gamma 100^{n+1}}{\alpha 3^n + \beta 5^n + \gamma 100^n} &
						\leq \lim_{n\to\infty} x_n &
						\leq \lim_{n\to\infty} \frac{\alpha 3^{n+1} + \beta 5^{n+1} + \gamma 100^{n+1}}{\gamma 100^n}
					\\
						\lim_{n\to\infty} \frac{\gamma 100}{\alpha{(\frac{3}{100})}^n + \beta{(\frac{5}{100})}^n + \gamma} &
						\leq \lim_{n\to\infty} x_n &
						\leq \lim_{n\to\infty} \frac{\alpha 3^{n+1}}{\gamma 100^n} + \frac{\beta 5^{n+1}}{\gamma 100^n} + \frac{\gamma 100^{n+1}}{\gamma 100^n}
					\\
						\lim_{n\to\infty} 100 &
						\leq \lim_{n\to\infty} x_n &
						\leq \lim_{n\to\infty} \frac{3\alpha}{\gamma} {(\frac{3}{100})}^n + \lim_{n\to\infty} \frac{5\beta}{\gamma} {(\frac{5}{100})}^n + \lim_{n\to\infty} 100
					\\						
						100 &\leq \lim_{n\to\infty} x_n &\leq 100
					\\
						\Rightarrow &\lim_{n\to\infty} x_n &= 100
					\end{eqnarray*}

				\item para $\gamma = 0$ y $\beta \neq 0$

					\begin{eqnarray*}
						\lim_{n\to\infty} \frac{\beta 5^{n+1}}{\alpha 3^n + \beta 5^n} &
						\leq \lim_{n\to\infty} \frac{\alpha 3^{n+1} + \beta 5^{n+1}}{\alpha 3^n + \beta 5^n} &
						\leq \lim_{n\to\infty} \frac{\alpha 3^{n+1} + \beta 5^{n+1}}{\beta 5^n}
					\\
						\lim_{n\to\infty} \frac{\beta 5}{\alpha{(\frac{3}{5})}^n + \beta} &
						\leq \lim_{n\to\infty} x_n &
						\leq \lim_{n\to\infty} \frac{\alpha 3^{n+1}}{\beta 5^n} + \frac{\beta 5^{n+1}}{\beta 5^n}
					\\
						\lim_{n\to\infty} 5 &
						\leq \lim_{n\to\infty} x_n &
						\leq \lim_{n\to\infty} \frac{3\alpha}{\beta} {(\frac{3}{5})}^n + \lim_{n\to\infty} 5
					\\
						5 &\leq \lim_{n\to\infty} x_n &\leq 5
					\\
						& \Rightarrow \lim_{n\to\infty} x_n &= 5
					\end{eqnarray*}

				\item para $\gamma = 0$, $\beta = 0$ y $\alpha \neq 0$

					$$
						\lim_{n\to\infty} \frac{\alpha 3^{n+1}}{\alpha 3^n}
						= \lim_{n\to\infty} 3
						= 3
					$$
\\
				\item para $\gamma = \beta = \alpha = 0$ la función no esta definida,
				y por lo tanto no nos interesa, porque suponemos que para todo $x0$, $x1$ existen
				$\alpha$, $\beta$ y $\gamma$ que cumplen con la fórmula cerrada
				de tal manera que alguno de ellos es distinto de $0$.

			\end{itemize}
 
			Luego queremos ver para que casos de entrada, o valores de $x0$ y $x1$,
			se dan las distintas opciones para $\alpha$, $\beta$ y $\gamma$.

			Consideramos que si $\gamma$ puede ser $0$ entonces lo es, y solamente en caso contrario $\gamma!=0$.

			\begin{itemize}

				\item para $\gamma = 0$

					Despejamos la fórmula cerrada para $x0$ y $x1$ y restando las ecuaciones obtenemos la igualdad

					\begin{displaymath}
						b( \frac{5-x_0}{x_0-3} - \frac{5}{3} \frac{5-x_1}{x_1-3} ) = 0
					\end{displaymath}

					de la cuál podemos deducir que $x0(x1-8) = -15$.

					Dentro de este caso existe otro caso especial en el cual
					$\beta = 0$ y $\alpha \neq 0$. Podemos ver que este caso solo aparece
					cuando $x0 = x1 = 3$.				

				\item El resto de las entradas cae en el caso $\gamma \neq 0$

%					Para este caso, divido tanto dividendo como divisor
%					de la fórmula cerrada por $\gamma$,
%					generando así una nueva expresión de la fórmula,
%					esta vez en funcion de 2 variables
%					$\alpha' = \frac{\alpha}{\gamma}$ y $\beta'= \frac{\beta}{\gamma}$

%					\begin{displaymath}
%						x_n \ =\ \frac{\alpha' 3^{n+1} + \beta' 5^{n+1} + 100^{n+1}}{\alpha' 3^n + \beta' 5^n + 100^n}
%					\end{displaymath}

			\end{itemize}
 
			Luego quedan resueltos los límitesde la \textit{Recurrencia de Müller}
			en función de las distintas entradas de la siguiente manera:
 
 			$$\lim_{n\to\infty} x_n =
				\begin{cases}
				3 & x0 = 3 \land x1 = 3\\
				5 & x0(x1-8) = -15\\
				100 & \mbox{sino}\\
				\end{cases}
 			$$
 
		\end{usubsection}

		\begin{usubsection}{Implementación}

			Luego el trabajo consistió en implementar
			la \textit{Recurrencia de Müller} en distintos lenguajes y plataformas,
			y calcular hasta un n-ésimo término la sucesión
			de tal manera que $|x_n-x_{n-1}|<\epsilon$
			donde $\epsilon$ es una diferencia entre términos sucesivos
			a partir de la cuál consideramos que la recurrencia
			ya ha alcanzado el límite. \\
\\
			Los resultados de ejecutar estos programas son comparados
			con los resultados teóricos de límite obtenidos,
			explicados en la seccion precedente.
			Así podemos obtener un valor del error relativo
			cometido por el algoritmo por consecuencia de las limitaciones
			de representación aritmética de las computadoras. \\
\\
			Los resultados obtenidos son analizados en la sección siguiente.

		\end{usubsection}		

	\end{usection}
	
	\begin{usection}{Resultados}

		En este primer gráfico representamos los pares $(x_0,x_1)$ para los cuales el limite teórico es 5 (línea azul),
		y el punto $(3,3)$ que es el único que tiene límite teórico 3.
		El resto del plano son los $(x_0,x_1)$ para los cuales el límite es 100.
		
		\begin{figure}[H]
			\centering
			\caption{Dominio del límite de la sucesión}
			\label{fig:dominio}
				\gras{lim5.png}{.6}
		\end{figure}

		A continuación graficamos los sucesivos valores de $x_n$ para $1\leq n\leq 20$ con $x_0=2$ y $x_1=8$, comparando el resultado exacto con el resultado de generar la sucesion con la fórmula recurrente usando C++ (con double de presicion).
		Se observa que a pesar de unos errores importantes el límite es el mismo.

		\begin{figure}[H]
			\centering
				\gras{fig_2_8_20.png}{.6}
%%			\caption{La recurrencia se acomoda al exacto}
%%			\label{fig:acomodo}	
		\end{figure}

		\begin{figure}[H]
			\centering
%			\caption{Dominio del límite de la sucesión}
%			\label{fig:comparaciones}
				\gras{fig_4_4-25_20.png}{.6}
		\end{figure}

		\begin{figure}[H]
			\centering
%			\caption{Dominio del límite de la sucesión}
%			\label{fig:comparaciones}
				\gras{fig_4_4-25_10.png}{.6}
		\end{figure}

		\begin{figure}[H]
			\centering
			\caption{Comparación de lenguajes}
			\label{fig:comparaciones}
				\gras{fig_comparaciones.png}{.6}
		\end{figure}

		\begin{figure}[H]
			\centering
			\caption{Divergencia de la recurrencia para una instancia}
			\label{fig:divergencia}
				\gras{fig_divergencia.png}{.6}
		\end{figure}

		\begin{figure}[H]
			\centering
			\caption{Iteraciones necesarias para llegar a límite 100}
			\label{fig:iteraciones}
				\gras{fig_iteraciones.png}{.3}
		\end{figure}

	\end{usection}
	
	\begin{usection}{Discusión}

		Estudiando un poco los datos obtenidos
		y conociendo los problemas de representación de números reales
		en los lenguajes utilizados, podemos dar algunas posibles explicaciones
		para los errores que vemos en los resultados. \\

		En el caso de que los valores de entrada cumplan la condición $C(x0,x1)$
		necesaria para converger a un número, podemos deducir que
		todos los términos de la recurrencia para esta entrada
		deben cumplir también con la condición $C(x_n,x_{n+1})$ ya que
		si tomamos $x_0 = x_n$ y $x_1 = x_{n+1}$ la recurrencia
		para estos valores de entrada también debe converger al mismo número.

		Merece atención entonces notar en la figura \ref{fig:dominio},
		que casi todo el dominio corresponde a instancias
		tales que $\lim_{n\to\infty} x_n = 100$. 

		Por lo tanto, si $x_0$ y $x_1$ cumplen con la condición $x0(x1-8) = -15$,
		todos los pares de términos sucesivos de la recurrencia
		deben también caer en los puntos representados
		por la línea azul en el gráfico. Aqui vemos que con un pequeño error,
		un término puede fácilmente caer fuera de esta línea
		en el espacio que corresponde al límite $100$, para luego
		continuar convergiendo a $100$ erróneamente por la propiedad antes mencionada.

		En cambio, la probabilidad de que un término de una instancia
		de la sucesión que deba converger a $100$, caiga exactamente
		sobre la línea azul por un pequeño error , parece casi imposible,
		por lo que un error en el cálculo del límite para estas instancias no es de esperar.

		En el caso de que los valores de entrada sean $x0 = x1 = 3$,
		la recurrencia llega a su límite de inmediato
		sin necesidad de calcular mas de un término,
		y dado que la probabilidad de que se cometa un error
		mayor que el criterio de parada en una sola cuenta es casi nulo
		(para un buen criterio de parada), la probabilidad de error
		para este caso de entrada también es muy baja.

		Por esto es que notamos varios errores en el cálculo de límite
		para recurrencias que deben converger a $5$, y que terminan
		convergiendo a $100$, pero no en aquellas que convergen a $3$ o $100$. \\

		Es importante también aclarar, que aunque pareciera que
		en el caso promedio casi no encontramos errores,
		ya que para casi toda la entrada el limite es $100$,
		esto no necesariamente es cierto ya que no sabemos nada
		sobre la distribución de los datos de entrada.
		Estos podrían caer inicialmente en su mayoría en el caso
		en que $\lim_{n\to\infty} x_n = 5$ y podríamos estar cometiendo
		un error para la mayoría de los casos de entrada.

		Además no sabemos el significado de riesgo que puede llegar a tener
		un error en estos casos, por lo que tampoco podemos dejar que
		el programa responda erróneamente a los mismos ya que
		por alguna razon el factor de riesgo podría ser muy grande para esta entrada,
		mas alla de que la probabilidad de que toque uno de estos casos sea muy baja.

		Por lo tanto el cálculo del límite mediante la recurrencia,
		con este algoritmo y estas limitaciones en la aritmética puede llegar a ser incluso inútil. \\
\\
		En la figura \ref{fig:comparaciones} podemos observar como se comportan
		los distintos lenguajes para algunos casos de entrada
		particulares arbitrarios tal que $\lim_{n\to\infty} x_n = 5$,
		ya que no tiene mucho sentido analizar las recurrencias
		para $\lim_{n\to\infty} x_n = 3$ o $100$,
		ya que vimos que en estos casos es muy fácil para el algoritmo
		converger correctamente.

		Como podemos apreciar, el formato float de C hace
		que el programa "explote" mucho antes que el implementado
		con el tipo double, ya que este último tiene el doble de precision
		por lo cuál los errores que se generan son mucho mas chicos,
		y el error que se arrastra también.

		Como podemos leer en la referencia de python,
		los numeros de punto flotante en este lenguaje
		son implementados con el tipo double en C, y como vemos en el gráfico,
		la similitud de las precisiones hace que ambos algoritmos
		"exploten" aproximadamente en el mismo lugar.
		Difieren un poco en comportamiento ya que el python
		puede compilar un algoritmo ligeramente distinto que el g++.

		Sobre el \textit{OpenOffice spreadsheet} no tenemos referencias sobre la precisión usada,
		pero vemos claramente que el comportamiento es exactamente el mismo que el de python,
		por lo que creemos que la aritmética es resuelta en python,
		ya que el programa además acepta macros en este lenguaje.
		Esto no necesariamente se traduce a todas las instancias del problema,
		pero podemos intuir al menos que la precisión que usa es la misma,
		o sea la \textit{double precision IEEE-754}\superref{ref:IEEE754}.

		Por último tenemos la implementación del algoritmo en python
		pero usando una librería numerica llamada \textit{decimal}\superref{ref:decimal},
		que nos permite elegir una precisión arbitraria para
		la representación en punto flotante.
		Vemos que con una precisión de 2000 dígitos significativos decimales,
		el $X_{20}$ nos sigue dando sobre el límite exacto.
		Y calculamos que para esta entrada y esta precisión
		el algoritmo "explota" recien cerca del término $X_{1400}$.
		Así, podemos acotar el criterio de parada de la recurrencia
		con un epsilon arbitrariamente chico, eligiendo una precisión
		suficientemente grande. Esto sin embargo nunca garantiza
		que el algoritmo a partir de un término "explote". \\
\\
		Otro aspecto importante a discutir es la elección
		del criterio de parada para los algoritmos
		que calculan los términos de la sucesión por recurrencia.
		si tomamos $|x_{n+1}-x_n| < \epsilon$ como criterio de parada,
		el valor de $\epsilon$ depende tanto de la estabilidad
		del algoritmo como del tipo de precisión aritmética usada.

		Si el valor de $\epsilon$ es muy grande, probablemente $x_n$
		no este tan cerca del límite como quisieramos para tomarlo como tal.
		si al contrario lo hacemos muy chico, el cálculo de la recurrencia
		será mas prolongado y probablemente el error que lleva acumulado
		haga que la recurrencia salte a otro espacio de solución antes de llegar al límite correcto.

		Un criterio de parada más adecuado sería tal vez
		cortar la recurrencia cuando suceda que
		$|x_{n+2}-x_{n+1}| > |x_{n+1}-x_n|$, o sea, cuando la sucesión
		comienza a alejarse del lugar adonde estaba convergiendo.

		También se puede mejorar un poco este criterio
		aplicando una cota mínima para decidir
		si la recurrencia comienza a diverger, de manera
		de cuidarse de los casos en que la sucesion parezca diverger
		simplemente por un error aritmético.

	\end{usection}
	
	\begin{usection}{Conclusiones}
		Analizamos el comportamiento de la recurrencia en distintos entornos, con distintas precisiones y con distintos parámetros.
		La principal conclusión que sacamos es que sin importar las condiciones bajo las cuales se realiza el experimento, la solución nunca es exacta.
		Esto se debe a que la representación de los números reales en la computadora es inexacta, porque el almacenamiento es finito.
		
		Como es notado en las secciones anteriores, la recurrencia tiende a un límite distinto a 100 sólo en los puntos de la forma $(t, -15/t + 8)$. Si calculamos uno a uno los términos de la recurrencia para estos puntos, observamos que todos los pares $(x_k,x_{k+1})$ tienen que ser también de la forma $(t, -15/t + 8)$, (y son cada vez más cercanos al punto $(5,5)$).
		Dada esta condición, vemos que el más mínimo error en alguno de los cálculos, y el siguiente par de valores cae afuera de la curva.
		Como la recurrencia involucra restas y divisiones, un error de representacion y/o de cálculo es esperable en prácticamente todos los casos.\footnote{Valores particulares como por ejemplo $(5,5)$ no tienen error porque las operaciones de la recurrencia siempre dan enteros}
		
		Incluso con precisión arbitraria de 2000 digitos decimales la sucesión tiene un error importantísimo para los $x_k$ con $k\geq 1400$.
		
	\end{usection}
	
	\begin{usection}{Apéndices}
		\begin{usubsection}{Apéndice A: Enunciado}
			\begin{centering}
\bf Laboratorio de M\'etodos Num\'ericos - Primer cuatrimestre 2010 \\
\bf Trabajo Pr\'actico N\'umero 3: <El TP del Mundial! \\
\end{centering}

\vskip 25pt
\hrule
\vskip 11pt

\textbf{Introducci\'on}

Nos encontramos en la m\'axima cita del f\'utbol mundial...~de robots.
Uno de los problemas m\'as importantes a resolver en este contexto es
predecir con la mayor anticipaci\'on posible la posici\'on futura de la 
pelota en funci\'on de su posici\'on en el pasado reciente. Sobre la base
de estas predicciones se coordinan los movimientos de los jugadores de
campo y, en el caso que nos ocupa ahora, la posici\'on del arquero cuando
existe peligro de gol.

Cuando la pelota se dirige hacia nuestro arco, es muy importante que
ubiquemos el arquero en la posici\'on exacta en la que la pelota 
cruzar\'a la l\'\i nea de gol, de manera que pueda interceptarla y evitar
la ca\'\i da de nuestra valla. El sistema de control de cada equipo
suministra informaci\'on en pasos discretos. En cada paso nuestras
c\'amaras de video determinan la posici\'on de la pelota, y debemos
indicarle la acci\'on a seguir al arquero: quedarse quieto, moverse 
hacia la izquierda o moverse hacia la derecha.

Los postes del arco est\'an ubicados en las coordenadas (-1,0) y (1,0),
y la l\'\i nea de gol es el segmento entre estos dos puntos. Se marca
un gol cuando la pelota cruza este segmento. Vistos desde arriba, la pelota
es un c\'\i rculo de 0.10 de radio y el arquero 
se representa mediante un segmento paralelo a la l\'\i nea de gol de 0.10 
de longitud y ubicado sobre la misma.
% es un jugador cuadrado de 0.10 de lado, 
% cuyo punto central est\'a ubicado sobre la l\'\i nea de gol. 
Inicialmente el punto central del arquero se encuentra en la
posici\'on (0,0), y en cada paso se le indica al arquero qu\'e acci\'on
debe tomar. Si se le indica un movimiento hacia alguno de los lados
(izquierda o derecha) y en el paso anterior estaba quieto o se estaba
moviendo hacia ese mismo lado, entonces el arquero se mueve 0.05 en
la direcci\'on indicada. Por el contrario, si en el paso anterior se
hab\'\i a indicado un movimiento en la direcci\'on opuesta, entonces
el arquero se queda quieto durante el paso actual.

Un problema fundamental que debe enfrentarse es la presencia de ruido
en las mediciones de la posici\'on de la pelota. El sistema de visi\'on
est\'a sujeto a vibraciones, golpes y errores de captura de datos,
que hacen que las mediciones de la pelota sufran errores, e incluso 
registren posiciones irreales (es un efecto muy com\'un que la pelota
``desaparezca'' en un cuadro y vuelva a aparecer en el cuadro siguiente).
Por otra parte, la pelota no siempre viaja hacia el arco en l\'\i nea
recta sino que puede describir curvas m\'as o menos complicadas,
dependiendo del ``efecto'' dado por el jugador al momento de impactar
la pelota y de posibles curvaturas en la superficie del campo de juego.

\textbf{Enunciado}

El objetivo del trabajo pr\'actico es implementar un programa que
tome como datos las posiciones sucesivas de la pelota y que determine
en cada paso qu\'e debe hacer el arquero para evitar el gol. Se deben
tomar las mediciones con la posici\'on de la pelota de un archivo
de entrada, que tiene en cada l\'inea el n\'umero de medici\'on,
la posici\'on $x$ y la posici\'on $y$ de la pelota, separados
por espacios. La \'ultima l\'\i nea del archivo tiene -1 como
primer dato, indicando el fin de las mediciones.

En cada paso, el programa debe determinar qu\'e acci\'on debe tomar
el arquero, escribiendo la decisi\'on correspondiente en un archivo
de salida. Cada l\'\i nea de este archivo debe tener el n\'umero
de medici\'on y la acci\'on del arquero (0: quedarse quieto, 1: izquierda,
2: derecha), separados por espacio. El programa debe tomar por
l\'\i nea de comandos el nombre del archivo de entrada y el nombre
del archivo de salida. Dado que estamos simulando
la decisi\'on en tiempo real, para generar la acci\'on correspondiente
a una medici\'on el programa solamente puede usar la informaci\'on
de esa medici\'on y las anteriores (es decir, no se puede consultar
lo que suceder\'a en el futuro para tomar las decisiones).

Las instrucciones al arquero deben estar basadas en un mecanismo de
predicci\'on de la posici\'on futura de la pelota. Esta predicci\'on
se debe realizar sobre la base de alg\'un m\'etodo num\'erico visto
en la materia, o alguna variaci\'on de los temas vistos en clase.
Sugerimos consultar con los docentes del laboratorio para validar
los enfoques que propongan implementar.

Se adjunta a este enunciado un programa simulador de los tiros al
arco, junto con algunos archivos de prueba para testear el formato
de los archivos de entrada y salida. Todos los programas participar\'an
de un campeonato mundial de arqueros, y el grupo cuyo arquero logre
atajar la mayor cantidad de tiros al arco se har\'a acreedor a la
copa ``Laboratorio de M\'etodos Num\'ericos'' al mejor guardavallas del
mundial.

\textbf{Preguntas adicionales}

\begin{enumerate}
\item >En qu\'e minuto del partido Argentina-Uni\'on Sovi\'etica del mundial
Italia '90 se lesion\'o Nery Pumpido, arquero de la selecci\'on
argentina?

\item >C\'omo se llamaba el \'arbitro del partido Argentina-Francia en
el mundial Argentina '78?

\item >Cu\'antos jugadores fueron expulsados en el mundial Italia '90
por derribar a Claudio Paul Caniggia?

\item >Cu\'antos pases hizo la selecci\'on argentina antes del gol de Maradona
ante Grecia en el mundial EEUU 94?

\item >Cu\'antos minutos jug\'o Ricardo Bochini en la semifinal del
mundial M\'exico '86?

\item >C\'omo se llamaba el arquero suplente de la selecci\'on argentina
en la final del mundial Uruguay '30?
\end{enumerate}

\vskip 15pt

\hrule

\vskip 11pt

Fecha de entrega: Viernes 25 de Junio

		\end{usubsection}
		
		\newpage
		\begin{usubsection}{Apéndice B: Codigos Fuente}
\textbf{Algoritmo para calcular $x_n$ de forma exacta, en Python}
\begin{verbatim}
def xn_exacto(x0,x1,n):
   # si gamma puede ser 0
   if abs((x1-8.)*x0 + 15.) < e:
      # elijo un b
      b = 1.
      # calculo a
      a = b*(5.-x0)/(x0-3.)
      return ( a*3.**(n+1) + b*5.**(n+1) ) / ( a*3.**n + b*5.**n )
   # si gamma != 0
   else:
      # calculo b/g
      bg = 50.*(600.-297.*x1+x0*(91.+2.*x1)) / (15.+x0*(x1-8.))
      #calculo a/g
      ag = (10000.-100.*x0-bg*(x0-5.)) / (x0-3.)
      return ( ag*3.**(n+1) + bg*5.**(n+1) + 100.**(n+1) ) /
             ( ag*3.**n + bg*5.**n + 100.**n )
\end{verbatim}

\bigskip
\bigskip
\textbf{Algoritmo que calcula la recurrencia, en Python con presicion arbitraria}
\begin{verbatim}
from decimal import *
getcontext().prec=2000    #presicion: 2000 digitos
x0,x1=tuple(map(Decimal,sys.argv[1:3]))
n=int(sys.argv[3])
for i in range(n):
   #hacer todas las cuentas en presicion arbitraria
   temp = Decimal(108)-(Decimal(815)-(Decimal(1500)/x0))/x1
   x0 = x1
   x1 = temp
   print x1
\end{verbatim}

\newpage
\textbf{Algoritmo que calcula la recurrencia, en C++}
\begin{verbatim}
double x0, x1, temp, e;
e = 1e-3;
cin >> x0 >> x1;
do{
   temp = 108.-(815.-(1500./x0))/x1;
   x0 = x1;
   x1 = temp;
}while( abs(x1-x0) >= e );
\end{verbatim}

\bigskip
\textbf{Función usada en la Hoja de cálculo para calcular los términos de la recurrencia}
\begin{verbatim}
=IF(ABS(B2-C2)<$B$1,C2,108-(815-(1500/B2))/C2)
\end{verbatim}
(\texttt{B1} es la celda donde esta almacenado $\epsilon$, \texttt{B2} y \texttt{C2} son $x_{n-2}$ y $x_{n-1}$ respectivamente)
		\end{usubsection}
	\end{usection}
	
	\begin{usection}{Referencias}

	\begin{enumerate}
		\item \label{ref:pythonfloat} \texttt{http://docs.python.org/tutorial/floatingpoint.html} \\
		Explicación de la aritmética de punto flotante en Python

		\item \label{ref:IEEE754} \texttt{http://steve.hollasch.net/cgindex/coding/ieeefloat.html} \\
		Estándard de representación de punto flotante más utilizado

		\item \label{ref:decimal} \texttt{http://docs.python.org/library/decimal.html}\\ 
		Documentación de la libreria \texttt{decimal} en Python, para presicion arbitraria
	\end{enumerate}	
	
	\end{usection}

\end{document}

