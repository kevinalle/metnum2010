\documentclass[12pt,titlepage]{article}
\usepackage[spanish]{babel}
\usepackage[utf8]{inputenc}
\usepackage{amsmath}
\usepackage{amssymb}
\usepackage{graphicx}
\usepackage{caratulaMetNum}
\usepackage{float}
\usepackage{subfigure}

 \usepackage{a4wide}
% \usepackage{amssymb}
% \usepackage{amsmath}
 \usepackage{enumerate}
% \usepackage[utf8]{inputenc}
% \usepackage[spanish]{babel}
 \parindent = 0 pt
 \parskip = 11 pt
% \usepackage[width=15.5cm, left=3cm, top=2.5cm, height= 24.5cm]{geometry}

\usepackage{color}
\usepackage{url}
\definecolor{lnk}{rgb}{0,0,0.4}
\usepackage[colorlinks=true,linkcolor=lnk,citecolor=blue,urlcolor=blue]{hyperref}

\newcommand{\func}[2]{\texttt{#1}(#2) :}
\newcommand{\tab}{\hspace*{2em}}
\newcommand{\FOR}{\textbf{for }}
\newcommand{\TO}{\textbf{ to }}
\newcommand{\IF}{\textbf{if }}
\newcommand{\WHILE}{\textbf{while }}
\newcommand{\THEN}{\textbf{then }}
\newcommand{\ELSE}{\textbf{else }}
\newcommand{\RET}{\textbf{return }}
\newcommand{\MOD}{\textbf{ \% }}
\newcommand{\OR}{\textbf{ or }}
\newcommand{\NOT}{\textbf{ not }}
\newcommand{\tOde}[1]{\tab \small{\mathcal{O}($#1$)}}
\newcommand{\Ode}[1]{\ensuremath{\small{\mathcal{O}\left(#1\right)}}}
\newcommand{\VSP}{\vspace*{3em}}
\newcommand{\Pa}{\vspace{5mm}}
\newenvironment{pseudo}{\noindent\begin{tabular}{ll}}{\end{tabular}\VSP}

\newenvironment{while}{\WHILE \\ \setlength{\leftmargin}{0em} }{}

\newcommand{\iif}{\Leftrightarrow}
\newcommand{\gra}[1]{\noindent\includegraphics[scale=.70]{#1}\\}
\newcommand{\gras}[2]{\noindent\includegraphics[scale=#2]{#1}\\}
\newcommand{\gram}[1]{\noindent\includegraphics[scale=.50]{#1}}
\newcommand{\dirmail}[1]{\normalsize{\texttt{#1}}}
\newenvironment{usection}[1]{\newpage\begin{section}*{#1}	\addcontentsline{toc}{section}{#1}}{\end{section}}
\newenvironment{usubsection}[1]{\begin{subsection}*{#1}	\addcontentsline{toc}{subsection}{#1}}{\end{subsection}}

\newcommand{\superref}[1]{\textsuperscript{\ref{#1}}}

%\title{{\sc\normalsize Métodos Numéricos}\\{\bf Trabajo Práctico Nº1}}
%\author{\begin{tabular}{lcr}Pablo Herrero & LU & \dirmail{pablodherrero@gmail.com}\\Thomas Fischer & 489/08 & \dirmail{tfischer@dc.uba.ar}\\Kevin Allekotte & 490/08 & \dirmail{kevinalle@gmail.com} \end{tabular}}
%\date{\VSP \normalsize{Abril 2010}}
\begin{document}

\materia{Métodos Numéricos}
\titulo{Trabajo Práctico Nº1}
\subtitulo{``Perdidos'' en el Pacífico}
%\grupo{Grupo x}
\integrante{Pablo Herrero}{332/07}{pablodherrero@gmail.com}
\integrante{Thomas Fischer}{489/08}{tfischer@dc.uba.ar}
\integrante{Kevin Allekotte}{490/08}{kevinalle@gmail.com}

\abstracto{
	El siguiente trabajo se propone la comparación entre distintos
	métodos de simulación analítica de sistemas gravitatorios
	de varios cuerpos. El objetivo final es calcular las coordenadas
	para tirar un proyectil el 12 de junio del año 2010 a las 11:00hs,
	UTC-3 desde la órbita del planeta Neptuno para que el mismo impacte
	en la Tierra.
}

\palabraClave{}

\begin{titlepage}
\maketitle
\end{titlepage}
\tableofcontents
\newpage

	\begin{usection}{Introducción teorica}

		En la mecánica classica o Newtoniana, el problema de lo $n$
		cuerpos es el problema de predecir el movimiento de cuerpos
		celestes en mutua interacción gravitatoria para cualquier
		instante en el futuro o deducir su movimiento en cualquier
		instante del pasado.

		La acción gravitatoria entre los cuerpos depende de las
		posiciones relativas entre ellos, por lo que las ecuaciones de
		interacción gravitatoria quedan definidas en términos de
		ecuaciones diferenciales.

		El problema que surge al plantear el problema con $3 \leq n$ es
		que las ecuaciones el sistema quedan definidas en función de
		demasiadas variables como para aplicar los métodos de integración
		conocidos hoy en día para resolver las ecuaciones.

		Luego tenemos que encontrar otra forma de calcular las
		variables. En nuestro caso elegimos el método de simulación,
		el cuál introduce errores analíticos por discretizar el tiempo y
		de redondeo por trabajar con la aritmética finita de la
		computadora, pero que tiene la ventaja que el sistema a resolver
		es lineal, para lo cuál conocemos un método.

		Para minimizar estos errores implementamos distintos métodos de
		cálculo numérico de aproximación que validamos con sistemas
		familiares como el sistema solar, del cual sabemos por ejemplo
		que las órbitas son elípticas y bastante estables. Si dibujamos
		las órbitas simuladas podemos ver si se comportan de manera
		similar a lo esperado. Además, tenemos datos que se supone son
		bastante precisos descargados de la web oficial de la NASA para
		comparar.

	\end{usection}

	\begin{usection}{Desarrollo}

		En primer lugar el desarrollo consistio en implementar módulos
		para las operaciones con y entre Matrices y Vectores que
		sirviesen para resolver sistemas lineales. Esto implicaba
		también implementar el algoritmo de eliminación Gaussiana con
		pivoteo parcial para resolver el sistema lineal del segundo
		método de resolución. Los módulos fueron implementados en el
		lenguaje \texttt{C++} para garantizar una buena performance.

		Luego procedimos a implementar un entorno de simulación
		(\texttt{C++}) y un pequeño script para plotear y poder
		visualizar los resultados (\texttt{Python}). 

		Con todo el backend funcionando, procedimos a implementar
		gradualmente los distintos algoritmos de solución de sistemas
		lineales, intercalando la producción de cada uno con fases de
		validación que consistían en simular los sistemas propuestos
		en el enunciado, y cuyos resultados se pueden observar en la
		sección RESULTADOS.

		Validados los métodos de simulación, tuvimos que idear un
		algoritmo para calcular el vector de velocidad inicial de
		nuestro misil dirigido a la tierra, de tal manera que le pegase.
		La idea fue hacer una especie de búsqueda local, ya que
		suponíamos que de no colisionar con otro planeta en la
		trayectoria hasta la tierra, si pasabamos con una trayectoría
		mas cerca de la tierra que con otra, podíamos reducir la
		búsqueda de la solución a un espacio localizado cerca de la
		primera. Con un espacio o una trayectoria nos referimos a una
		velocidad inicial, la que en nuestro código esta determinada por
		dos ángulos de disparo. Uno sobre el eje $z$ y otro sobre el
		plano $xy$.

		para cada trayectoria inicial, definimos una grilla de $3x3$
		puntos, donde el punto central era ella misma, y los otros
		puntos son las rotaciones de la velocidad sobre los ejes
		anteriormente mencionados por un ángulo que se reduce a la
		mitad en cada iteración de la búsqueda, y luego disparámos
		misiles en todas las 9 direcciónes.
		Así, aunque en algunos casos asintóticamente,
		revisamos todo el espacio contenido en la grilla de la primera
		iteración, a la cuál le ponemos un ancho inicial grosero, de
		manera de incluír necesariamente a una solución.
		En la imágencontinuación se puede ver como evoluciona la grilla
		de la búsqueda en las sucesivas iteraciones.

		\begin{figure}[H]
			\centering
			\includegraphics[scale=0.5]{img/grid.png}
			\caption{
				En la imagen vemos una representación gráfica de los
				sucesivos espacios de búsqueda local. el orden de las
				iteraciones es rojo $\rightarrow$ azul
				$\rightarrow$ verde.
			}
			\label{fig:grilla}
		\end{figure}		

		

	\end{usection}

	\begin{usection}{Resultados}

		%\begin{figure}
	\centering
	\subfigure[$\Delta t$ = 1 día]{
	\includegraphics[scale=0.38]{img/ej1/metodo1/validacion_30_1.png}
	\label{fig:ej1_m1_30_1}
	}
	\subfigure[$\Delta t$ = 6 horas]{
	\includegraphics[scale=0.38]{img/ej1/metodo1/validacion_30_4.png}
	\label{fig:ej1_m1_30_4}
	}
	\\
	\subfigure[$\Delta t$ = 2 horas]{
	\includegraphics[scale=0.38]{img/ej1/metodo1/validacion_30_12.png}
	\label{fig:ej1_m1_30_12}
	}
	\subfigure[$\Delta t$ = 1 hora]{
	\includegraphics[scale=0.38]{img/ej1/metodo1/validacion_30_24.png}
	\label{fig:ej1_m1_30_24}
	}
	\caption{
		Simulación de validación del sistema sol$-$tierra$-$luna para un período de 30 días y distintos $\Delta t$
		con el método 1.
		Observamos que el error no parece ser tan grande a simple vista para este período.
	}
	\label{ fig:res_ej1_m1_30 }
\end{figure}
\begin{figure}
	\centering
	\subfigure[$\Delta t$ = 1 día]{
	\includegraphics[scale=0.38]{img/ej1/metodo1/validacion_365_1.png}
	\label{fig:ej1_m1_365_1}
	}
	\subfigure[$\Delta t$ = 6 horas]{
	\includegraphics[scale=0.38]{img/ej1/metodo1/validacion_365_4.png}
	\label{fig:ej1_m1_365_4}
	}
	\\
	\subfigure[$\Delta t$ = 2 horas]{
	\includegraphics[scale=0.38]{img/ej1/metodo1/validacion_365_12.png}
	\label{fig:ej1_m1_365_12}
	}
	\subfigure[$\Delta t$ = 1 hora]{
	\includegraphics[scale=0.38]{img/ej1/metodo1/validacion_365_24.png}
	\label{fig:ej1_m1_365_24}
	}
	\caption{
		Simulación de validación del sistema sol$-$tierra$-$luna para un período de 1 año y distintos $\Delta t$
		con el método 1.
		Para esta cantidad de tiempo, la simulación con un $\Delta t$ de un día ya merece ser descartada.
		Las otras todavía parecen comportarse razonablemente
	}
	\label{ fig:res_ej1_m1_365 }
\end{figure}
\begin{figure}
	\centering
	\subfigure[$\Delta t$ = 1 día]{
	\includegraphics[scale=0.38]{img/ej1/metodo1/validacion_1825_1.png}
	\label{fig:ej1_m1_1825_1}
	}
	\subfigure[$\Delta t$ = 6 horas]{
	\includegraphics[scale=0.38]{img/ej1/metodo1/validacion_1825_4.png}
	\label{fig:ej1_m1_1825_4}
	}
	\\
	\subfigure[$\Delta t$ = 2 horas]{
	\includegraphics[scale=0.38]{img/ej1/metodo1/validacion_1825_12.png}
	\label{fig:ej1_m1_1825_12}
	}
	\subfigure[$\Delta t$ = 1 hora]{
	\includegraphics[scale=0.38]{img/ej1/metodo1/validacion_1825_24.png}
	\label{fig:ej1_m1_1825_24}
	}
	\caption{
		Simulación de validación del sistema sol$-$tierra$-$luna para un período de 5 años y distintos $\Delta t$
		con el método 1.
		Para esta cantidad de tiempo de simulación, las órbitas de los planetas ya comienzan a mostrar un movimiento espiral en vez de elíptico más de una vuelta.
		El mejor método parece ser el de $\Delta t$ de 2 horas, ya que para el de 1 hora la luna comienza a comportarse de una manera un poco mas extraña,
		comportamiento que atribuímos a errores de redondeo por tratar con numeros muy chicos o muy grandes al multiplicar o dividir respectivamente por un $\Delta t$ muy chico.
	}
	\label{ fig:res_ej1_m1_1825 }
\end{figure}

		%\begin{figure}
	\centering
	\subfigure[$\Delta t$ = 1 día]{
	\includegraphics[scale=0.38]{img/ej1/metodo2/validacion_30_1.png}
	\label{fig:ej1_m2_30_1}
	}
	\subfigure[$\Delta t$ = 6 horas]{
	\includegraphics[scale=0.38]{img/ej1/metodo2/validacion_30_4.png}
	\label{fig:ej1_m2_30_4}
	}
	\\
	\subfigure[$\Delta t$ = 2 horas]{
	\includegraphics[scale=0.38]{img/ej1/metodo2/validacion_30_12.png}
	\label{fig:ej1_m2_30_12}
	}
	\subfigure[$\Delta t$ = 1 hora]{
	\includegraphics[scale=0.38]{img/ej1/metodo2/validacion_30_24.png}
	\label{fig:ej1_m2_30_24}
	}
	\caption{
		Simulación de validación del sistema sol$-$tierra$-$luna para un período de 30 días y distintos $\Delta t$
		con el método 2.
		Al parecer el método 2 tiende a ser mas sensible a los errores por discretizacion del tiempo que el 1,
		ya que para un intervalo de tiempo pequeño, como lo es 1 mes,
		notamos cambios importantes en la precision con un $\Delta t$ grande (1 día) con respecto al método 1.
	}
	\label{ fig:res_ej1_m2_30 }
\end{figure}
\begin{figure}
	\centering
	\subfigure[$\Delta t$ = 1 día]{
	\includegraphics[scale=0.38]{img/ej1/metodo2/validacion_365_1.png}
	\label{fig:ej1_m2_365_1}
	}
	\subfigure[$\Delta t$ = 6 horas]{
	\includegraphics[scale=0.38]{img/ej1/metodo2/validacion_365_4.png}
	\label{fig:ej1_m2_365_4}
	}
	\\
	\subfigure[$\Delta t$ = 2 horas]{
	\includegraphics[scale=0.38]{img/ej1/metodo2/validacion_365_12.png}
	\label{fig:ej1_m2_365_12}
	}
	\subfigure[$\Delta t$ = 1 hora]{
	\includegraphics[scale=0.38]{img/ej1/metodo2/validacion_365_24.png}
	\label{fig:ej1_m2_365_24}
	}
	\caption{
		Simulación de validación del sistema sol$-$tierra$-$luna para un período de 1 año y distintos $\Delta t$
		con el método 2.
		Para este método en este tiempo de simulación ya podemos descartar el $\Delta t$ de 6 horas como simulación razonable,
		cuando con el método 1 para este período de tiempo solo descartabamos el $\Delta t$ de 1 día.
		Sigue manifestándose entonces la alta sensibilidad a los errores por discretizacion del tiempo.
	}
	\label{ fig:res_ej1_m2_365 }
\end{figure}
\begin{figure}
	\centering
	\subfigure[$\Delta t$ = 1 día]{
	\includegraphics[scale=0.38]{img/ej1/metodo2/validacion_1825_1.png}
	\label{fig:ej1_m2_1825_1}
	}
	\subfigure[$\Delta t$ = 6 horas]{
	\includegraphics[scale=0.38]{img/ej1/metodo2/validacion_1825_4.png}
	\label{fig:ej1_m2_1825_4}
	}
	\\
	\subfigure[$\Delta t$ = 2 horas]{
	\includegraphics[scale=0.38]{img/ej1/metodo2/validacion_1825_12.png}
	\label{fig:ej1_m2_1825_12}
	}
	\subfigure[$\Delta t$ = 1 hora]{
	\includegraphics[scale=0.38]{img/ej1/metodo2/validacion_1825_24.png}
	\label{fig:ej1_m2_1825_24}
	}
	\caption{
		Simulación de validación del sistema sol$-$tierra$-$luna para un período de 5 años y distintos $\Delta t$
		con el método 2.
		Acá ya vemos que el único $\Delta t$ razonable que queda es el de 1 hora,
		pero vale la pena notar que es el mejor de los resultados obtenidos hasta ahora en este período de tiempo.
	}
	\label{ fig:res_ej1_m2_1825 }
\end{figure}

		%\begin{figure}
	\centering
	\subfigure[$\delta t$ = 1 día]{
	\includegraphics[scale=0.38]{img/ej2/metodo_1/validacion_30_1.png}
	\label{fig:ej2_m1_30_1}
	}
	\subfigure[$\delta t$ = 6 horas]{
	\includegraphics[scale=0.38]{img/ej2/metodo_1/validacion_30_4.png}
	\label{fig:ej2_m1_30_4}
	}
	\\
	\subfigure[$\delta t$ = 2 horas]{
	\includegraphics[scale=0.38]{img/ej2/metodo_1/validacion_30_12.png}
	\label{fig:ej2_m1_30_12}
	}
	\subfigure[$\delta t$ = 1 hora]{
	\includegraphics[scale=0.38]{img/ej2/metodo_1/validacion_30_24.png}
	\label{fig:ej2_m1_30_24}
	}
	\caption{
		Simulación de validación del sistema solar para un período de 30 días y distintos $\delta t$
		con el método 1.
		Observamos que el error no parece ser tan grande a simple vista para este período.
	}
	\label{ fig:res_ej2_m1_30 }
\end{figure}
\begin{figure}
	\centering
	\subfigure[$\delta t$ = 1 día]{
	\includegraphics[scale=0.38]{img/ej2/metodo_1/validacion_365_1.png}
	\label{fig:ej2_m1_365_1}
	}
	\subfigure[$\delta t$ = 6 horas]{
	\includegraphics[scale=0.38]{img/ej2/metodo_1/validacion_365_4.png}
	\label{fig:ej2_m1_365_4}
	}
	\\
	\subfigure[$\delta t$ = 2 horas]{
	\includegraphics[scale=0.38]{img/ej2/metodo_1/validacion_365_12.png}
	\label{fig:ej2_m1_365_12}
	}
	\subfigure[$\delta t$ = 1 hora]{
	\includegraphics[scale=0.38]{img/ej2/metodo_1/validacion_365_24.png}
	\label{fig:ej2_m1_365_24}
	}
	\caption{
		Simulación de validación del sistema solar para un período de 1 año y distintos $\delta t$
		con el método 1.
		Para esta cantidad de tiempo, la simulación con un $\delta t$ de un día ya merece ser descartada.
		Las otras todavía parecen comportarse razonablemente
	}
	\label{ fig:res_ej2_m1_365 }
\end{figure}
\begin{figure}
	\centering
	\subfigure[$\delta t$ = 1 día]{
	\includegraphics[scale=0.38]{img/ej2/metodo_1/validacion_1825_1.png}
	\label{fig:ej2_m1_1825_1}
	}
	\subfigure[$\delta t$ = 6 horas]{
	\includegraphics[scale=0.38]{img/ej2/metodo_1/validacion_1825_4.png}
	\label{fig:ej2_m1_1825_4}
	}
	\\
	\subfigure[$\delta t$ = 2 horas]{
	\includegraphics[scale=0.38]{img/ej2/metodo_1/validacion_1825_12.png}
	\label{fig:ej2_m1_1825_12}
	}
	\subfigure[$\delta t$ = 1 hora]{
	\includegraphics[scale=0.38]{img/ej2/metodo_1/validacion_1825_24.png}
	\label{fig:ej2_m1_1825_24}
	}
	\caption{
		Simulación de validación del sistema solar para un período de 5 años y distintos $\delta t$
		con el método 1.
		Para esta cantidad de tiempo de simulación, las órbitas de los planetas ya comienzan a mostrar error al dar más de una vuelta.
		El mejor método parece ser el de $\delta t$ de 2 horas, ya que para el de 1 hora la luna comienza a comportarse de una manera un poco mas extraña.
	}
	\label{ fig:res_ej2_m1_1825 }
\end{figure}

		%\begin{figure}
	\centering
	\subfigure[$\Delta t$ = 1 día]{
	\includegraphics[scale=0.38]{img/ej2/metodo_2/validacion_30_1.png}
	\label{fig:ej2_m2_30_1}
	}
	\subfigure[$\Delta t$ = 6 horas]{
	\includegraphics[scale=0.38]{img/ej2/metodo_2/validacion_30_4.png}
	\label{fig:ej2_m2_30_4}
	}
	\\
	\subfigure[$\Delta t$ = 2 horas]{
	\includegraphics[scale=0.38]{img/ej2/metodo_2/validacion_30_12.png}
	\label{fig:ej2_m2_30_12}
	}
	\subfigure[$\Delta t$ = 1 hora]{
	\includegraphics[scale=0.38]{img/ej2/metodo_2/validacion_30_24.png}
	\label{fig:ej2_m2_30_24}
	}
	\caption{
		Simulación de validación del sistema solar para un período de 30 días y distintos $\Delta t$
		con el método 2.
		Observamos que el error no parece ser tan grande a simple vista para este período.
	}
	\label{ fig:res_ej2_m2_30 }
\end{figure}
\begin{figure}
	\centering
	\subfigure[$\Delta t$ = 1 día]{
	\includegraphics[scale=0.38]{img/ej2/metodo_2/validacion_365_1.png}
	\label{fig:ej2_m2_365_1}
	}
	\subfigure[$\Delta t$ = 6 horas]{
	\includegraphics[scale=0.38]{img/ej2/metodo_2/validacion_365_4.png}
	\label{fig:ej2_m2_365_4}
	}
	\\
	\subfigure[$\Delta t$ = 2 horas]{
	\includegraphics[scale=0.38]{img/ej2/metodo_2/validacion_365_12.png}
	\label{fig:ej2_m2_365_12}
	}
	\subfigure[$\Delta t$ = 1 hora]{
	\includegraphics[scale=0.38]{img/ej2/metodo_2/validacion_365_24.png}
	\label{fig:ej2_m2_365_24}
	}
	\caption{
		Simulación de validación del sistema solar para un período de 1 año y distintos $\Delta t$
		con el método 2.
		Otra vez no hay mucho que ver para este período de tiempo.
	}
	\label{ fig:res_ej2_m2_365 }
\end{figure}
\begin{figure}
	\centering
	\subfigure[$\Delta t$ = 1 día]{
	\includegraphics[scale=0.38]{img/ej2/metodo_2/validacion_1825_1.png}
	\label{fig:ej2_m2_1825_1}
	}
	\subfigure[$\Delta t$ = 6 horas]{
	\includegraphics[scale=0.38]{img/ej2/metodo_2/validacion_1825_4.png}
	\label{fig:ej2_m2_1825_4}
	}
	\\
	\subfigure[$\Delta t$ = 2 horas]{
	\includegraphics[scale=0.35]{img/ej2/metodo_2/validacion_1825_12.png}
	\label{fig:ej2_m2_1825_12}
	}
	\subfigure[$\Delta t$ = 1 hora]{
	\includegraphics[scale=0.35]{img/ej2/metodo_2/validacion_1825_24.png}
	\label{fig:ej2_m2_1825_24}
	}
	\caption{
		Simulación de validación del sistema solar para un período de 5 años y distintos $\Delta t$
		con el método 2.
		Aquí vemos de nuevo lo sensible que es el segundo método a los errores producidos por la discretización del tiempo.
		Vemos como para mercurio se pierde totalmente la órbita
		(probablemente la causa es la proximidad de un cuerpo con mucha masa como el sol)
		y este sale disparado fuera del sistema. Al simular con mas definición, sin embargo, el producto mejora notablemente.
	}
	\label{ fig:res_ej2_m2_1825 }
\end{figure}


	\end{usection}
	
	\begin{usection}{Discusión}
		
	\end{usection}
	
	\begin{usection}{Conclusiones}
		
	\end{usection}
	
	\begin{usection}{Apéndices}
		\begin{usubsection}{Apéndice A: Enunciado}
			\begin{centering}
\bf Laboratorio de M\'etodos Num\'ericos - Primer cuatrimestre 2010 \\
\bf Trabajo Pr\'actico N\'umero 3: <El TP del Mundial! \\
\end{centering}

\vskip 25pt
\hrule
\vskip 11pt

\textbf{Introducci\'on}

Nos encontramos en la m\'axima cita del f\'utbol mundial...~de robots.
Uno de los problemas m\'as importantes a resolver en este contexto es
predecir con la mayor anticipaci\'on posible la posici\'on futura de la 
pelota en funci\'on de su posici\'on en el pasado reciente. Sobre la base
de estas predicciones se coordinan los movimientos de los jugadores de
campo y, en el caso que nos ocupa ahora, la posici\'on del arquero cuando
existe peligro de gol.

Cuando la pelota se dirige hacia nuestro arco, es muy importante que
ubiquemos el arquero en la posici\'on exacta en la que la pelota 
cruzar\'a la l\'\i nea de gol, de manera que pueda interceptarla y evitar
la ca\'\i da de nuestra valla. El sistema de control de cada equipo
suministra informaci\'on en pasos discretos. En cada paso nuestras
c\'amaras de video determinan la posici\'on de la pelota, y debemos
indicarle la acci\'on a seguir al arquero: quedarse quieto, moverse 
hacia la izquierda o moverse hacia la derecha.

Los postes del arco est\'an ubicados en las coordenadas (-1,0) y (1,0),
y la l\'\i nea de gol es el segmento entre estos dos puntos. Se marca
un gol cuando la pelota cruza este segmento. Vistos desde arriba, la pelota
es un c\'\i rculo de 0.10 de radio y el arquero 
se representa mediante un segmento paralelo a la l\'\i nea de gol de 0.10 
de longitud y ubicado sobre la misma.
% es un jugador cuadrado de 0.10 de lado, 
% cuyo punto central est\'a ubicado sobre la l\'\i nea de gol. 
Inicialmente el punto central del arquero se encuentra en la
posici\'on (0,0), y en cada paso se le indica al arquero qu\'e acci\'on
debe tomar. Si se le indica un movimiento hacia alguno de los lados
(izquierda o derecha) y en el paso anterior estaba quieto o se estaba
moviendo hacia ese mismo lado, entonces el arquero se mueve 0.05 en
la direcci\'on indicada. Por el contrario, si en el paso anterior se
hab\'\i a indicado un movimiento en la direcci\'on opuesta, entonces
el arquero se queda quieto durante el paso actual.

Un problema fundamental que debe enfrentarse es la presencia de ruido
en las mediciones de la posici\'on de la pelota. El sistema de visi\'on
est\'a sujeto a vibraciones, golpes y errores de captura de datos,
que hacen que las mediciones de la pelota sufran errores, e incluso 
registren posiciones irreales (es un efecto muy com\'un que la pelota
``desaparezca'' en un cuadro y vuelva a aparecer en el cuadro siguiente).
Por otra parte, la pelota no siempre viaja hacia el arco en l\'\i nea
recta sino que puede describir curvas m\'as o menos complicadas,
dependiendo del ``efecto'' dado por el jugador al momento de impactar
la pelota y de posibles curvaturas en la superficie del campo de juego.

\textbf{Enunciado}

El objetivo del trabajo pr\'actico es implementar un programa que
tome como datos las posiciones sucesivas de la pelota y que determine
en cada paso qu\'e debe hacer el arquero para evitar el gol. Se deben
tomar las mediciones con la posici\'on de la pelota de un archivo
de entrada, que tiene en cada l\'inea el n\'umero de medici\'on,
la posici\'on $x$ y la posici\'on $y$ de la pelota, separados
por espacios. La \'ultima l\'\i nea del archivo tiene -1 como
primer dato, indicando el fin de las mediciones.

En cada paso, el programa debe determinar qu\'e acci\'on debe tomar
el arquero, escribiendo la decisi\'on correspondiente en un archivo
de salida. Cada l\'\i nea de este archivo debe tener el n\'umero
de medici\'on y la acci\'on del arquero (0: quedarse quieto, 1: izquierda,
2: derecha), separados por espacio. El programa debe tomar por
l\'\i nea de comandos el nombre del archivo de entrada y el nombre
del archivo de salida. Dado que estamos simulando
la decisi\'on en tiempo real, para generar la acci\'on correspondiente
a una medici\'on el programa solamente puede usar la informaci\'on
de esa medici\'on y las anteriores (es decir, no se puede consultar
lo que suceder\'a en el futuro para tomar las decisiones).

Las instrucciones al arquero deben estar basadas en un mecanismo de
predicci\'on de la posici\'on futura de la pelota. Esta predicci\'on
se debe realizar sobre la base de alg\'un m\'etodo num\'erico visto
en la materia, o alguna variaci\'on de los temas vistos en clase.
Sugerimos consultar con los docentes del laboratorio para validar
los enfoques que propongan implementar.

Se adjunta a este enunciado un programa simulador de los tiros al
arco, junto con algunos archivos de prueba para testear el formato
de los archivos de entrada y salida. Todos los programas participar\'an
de un campeonato mundial de arqueros, y el grupo cuyo arquero logre
atajar la mayor cantidad de tiros al arco se har\'a acreedor a la
copa ``Laboratorio de M\'etodos Num\'ericos'' al mejor guardavallas del
mundial.

\textbf{Preguntas adicionales}

\begin{enumerate}
\item >En qu\'e minuto del partido Argentina-Uni\'on Sovi\'etica del mundial
Italia '90 se lesion\'o Nery Pumpido, arquero de la selecci\'on
argentina?

\item >C\'omo se llamaba el \'arbitro del partido Argentina-Francia en
el mundial Argentina '78?

\item >Cu\'antos jugadores fueron expulsados en el mundial Italia '90
por derribar a Claudio Paul Caniggia?

\item >Cu\'antos pases hizo la selecci\'on argentina antes del gol de Maradona
ante Grecia en el mundial EEUU 94?

\item >Cu\'antos minutos jug\'o Ricardo Bochini en la semifinal del
mundial M\'exico '86?

\item >C\'omo se llamaba el arquero suplente de la selecci\'on argentina
en la final del mundial Uruguay '30?
\end{enumerate}

\vskip 15pt

\hrule

\vskip 11pt

Fecha de entrega: Viernes 25 de Junio

		\end{usubsection}
		
		\newpage
		\begin{usubsection}{Apéndice B: Codigos Fuente}
			
		\end{usubsection}
	\end{usection}
	
	\begin{usection}{Referencias}

	\begin{enumerate}
		\item \label{ref:label} \texttt{http://url} \\ Descripcion
	\end{enumerate}	
	
	\end{usection}

\end{document}

